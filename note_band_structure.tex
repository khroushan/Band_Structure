\documentclass[11pt]{article}
\usepackage{amsmath}
%%== Packages ==%%

%%== New command ==%%
\newcommand{\hl}{\mathcal{H}}
\newcommand{\rg}{\rangle}
\newcommand{\leg}{\langle}
\newcommand{\ep}{\epsilon}
%===========================
\author{Amin Ahmadi}
\date{16 May 2016}
\title{Band structure}
\begin{document}
\section{Square Lattice}
A 1D chain lattice in tight-binding approximation has the
Hamiltonian of the form
\begin{equation}
  \hl = \sum_{n=-\infty}^{\infty} \epsilon_0 |n\rg\leg n|
  -t\sum_{n=-\infty}^\infty (|n\rg\leg n+1| + |n+1\rg\leg n|)
\end{equation}
where $t$ is the hopping amplitude and $\ep_0$ is the
on-site energy. The matrix representation of the Hamiltonian
reads
\begin{equation}
  \hl =
  \begin{pmatrix}
    \ep_0 & -t & 0 & \ldots 0 \\
    -t & \ep_0 & -t & 0 & \ldots \\
    \vdots
  \end{pmatrix}
\end{equation}
to compute the dispertion relation $E=E(k)$ one can use the
fact that the physical properties of the lattice is
invariant under translational transformation, means
\begin{equation}
  [\hl,\tau(a)] = 0
\end{equation}
where $a$ is the lattice distance. At this point we are
looking for an approach to transform the eigenvalue problem
of $\infty\times\infty$ Hamiltonian to $m\times m$ problem
where $m$ is the dimension of a single unit cell.

Assume each unit cell is seperated with distance $a$ and has
an internal staructure which can be described with the
Hamiltonian matrix $h$. The hopping matrix is $\tau$. In
this manner the total Hamiltonian has the form of
\begin{equation}
  \hl = \sum_{n=-\infty}^{\infty} h |n\rg\leg n|
  -\sum_{n=-\infty}^\infty (\tau|n\rg\leg n+1| + \tau^\dag|n+1\rg\leg n|)
\end{equation} 
where $h$ and $u$ matrices are $m\time m$ dimensional which
is the number of sites in the unit cell. The band structure
problem which is the eigen-energies of whole lattice can be
reduced to the eigenvalue problem of unit cell using the
fact 
\begin{equation}
  \hat{u}\phi_{n+1} = \hat{u} e^{ika} \phi_n 
\end{equation}
\end{document}
